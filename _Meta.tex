% Informationen ------------------------------------------------------------
%   Definition von globalen Parametern, die im gesamten Dokument verwendet
%   werden knnen (z.B auf dem Deckblatt etc.).
% --------------------------------------------------------------------------
\newcommand{\titel}{Erstellen einer Android App für die Esport-Webseite readmore.de}
\newcommand{\untertitel}{Leer}
\newcommand{\art}{Studienarbeit}
\newcommand{\fachgebiet}{Angewandte Informatik}
\newcommand{\autor}{Tobias Bechtold}
\newcommand{\matrikelnr}{6463863}
\newcommand{\studienbereich}{Angewandte Informatik}
\newcommand{\kursbez}{TINF12B2}
\newcommand{\firmenname}{Fiducia IT AG}
\newcommand{\jahr}{2015}
\newcommand{\abgabedatum}{30. Mai 2015}

% Eigene Befehle
\newcommand{\todo}[1]{\textbf{\textsc{\textcolor{red}{(TODO: #1)}}}}

% Autorennamen in small caps
\newcommand{\AutorZ}[1]{\textsc{#1}}
\newcommand{\Autor}[1]{\AutorZ{\citeauthor{#1}}}

% Befehle zur semantischen Auszeichnung von Text
\newcommand{\NeuerBegriff}[1]{\textbf{#1}}
\newcommand{\Fachbegriff}[1]{\textit{#1}}
\newcommand{\Prozess}[1]{\textit{#1}}
\newcommand{\Webservice}[1]{\textit{#1}}
\newcommand{\Eingabe}[1]{\texttt{#1}}
\newcommand{\Code}[1]{\texttt{#1}}
\newcommand{\Datei}[1]{\texttt{#1}}
\newcommand{\Datentyp}[1]{\textsf{#1}}
\newcommand{\XMLElement}[1]{\textsf{#1}}

% Abkuerzungen
\newcommand{\vgl}{Vgl.\ }
\newcommand{\ua}{\mbox{u.\,a.\ }}
\newcommand{\zB}{\mbox{z.\,B.\ }}
\newcommand{\bs}{$\backslash$}
