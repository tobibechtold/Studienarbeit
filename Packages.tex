% Anpassung des Seitenlayouts ----------------------------------------------
%   siehe Seitenstil.tex
% --------------------------------------------------------------------------
\usepackage[
  automark,     % Kapitelangaben in Kopfzeile automatisch erstellen
  headsepline,  % Trennlinie unter Kopfzeile
  ilines        % Trennlinie linksbndig ausrichten
]{scrpage2}


% Anpassung an Landessprache -----------------------------------------------
%   Verwendet globale Option german siehe \documentclass
% --------------------------------------------------------------------------
\usepackage{babel}

\usepackage[printonlyused]{acronym}

% Grafiken -----------------------------------------------------------------
%     Einbinden von Grafiken [draft oder final]
%     Option [draft] bindet Bilder nicht ein - auch globale Option
% --------------------------------------------------------------------------
\usepackage[dvips,final]{graphicx}
\graphicspath{{Bilder/}} % Dort liegen die Bilder des Dokuments


% Fuer Index-Ausgabe; \printindex -------------------------------------------
\usepackage{makeidx}

% Einfache Definition der Zeilenabstaende und Seitenrnder etc. -------------
\usepackage{setspace}
\usepackage{geometry}


\usepackage{xcolor} 
\definecolor{hellgelb}{rgb}{1,1,0.9}
\definecolor{colKeys}{rgb}{0,0,1}
\definecolor{colIdentifier}{rgb}{0,0,0}
\definecolor{colComments}{rgb}{1,0,0}
\definecolor{colString}{rgb}{0,0.5,0}

% Lange URLs umbrechen etc. -------------------------------------------------
\usepackage{url}

\usepackage{footnote} % Ermoeglicht Fussnoten in gleitenden Umgebungen
\makesavenoteenv[figure*]{figure}

\pdfcompresslevel=1
%\pdfimageresolution=1200
%\pdfpkresolution=1200

% Zum fortlaufenden Durchnummerieren der Fussnoten ---------------------------
\usepackage{chngcntr}

\usepackage{caption}

\usepackage{tocloft}

\newcommand{\myappendix}[1]{%
  \refstepcounter{appendix}%
  \section*{\theappendix\space #1}%
  \addcontentsline{app}{appendix}{\protect\numberline{\theappendix}#1}%
  \par
}

\newcommand{\subappendix}[1]{%
  \refstepcounter{subappendix}%
  \subsection*{\thesubappendix\space #1}%
  \addcontentsline{app}{subappendix}{\protect\numberline{\thesubappendix}#1}%
}

% Formatierung von Listen ndern
\usepackage{paralist}
% Standardeinstellungen:
% \setdefaultleftmargin{2.5em}{2.2em}{1.87em}{1.7em}{1em}{1em}

%Zeilenabstand
\renewcommand{\baselinestretch}{1.5}\normalsize

%Schriftart
\renewcommand{\familydefault}{\sfdefault}


\usepackage[see, commabeforerest, authorformat=year]{jurabib}
\renewcommand*{\bibbtasep}{; } % bta = between two authors sep
\renewcommand*{\bibbfsasep}{; } % bfsa = between first and second author sep
\renewcommand*{\bibbstasep}{; }% bsta = between second and third author sep
\renewcommand*{\bibatsep}{, }
\DeclareRobustCommand{\jbaensep}{,}

% PDF-Optionen --------------------------------------------------------------
\usepackage[
bookmarks,
bookmarksopen=true,
pdftitle={\titel},
pdfauthor={\autor},
pdfcreator={\autor},
pdfsubject={\titel},
pdfkeywords={\titel},
pdfmenubar=true,
colorlinks=true,
linkcolor=black, % einfache interne Verknpfungen
anchorcolor=black,% Ankertext
citecolor=black, % Verweise auf Literaturverzeichniseintrge im Text
filecolor=black, % Verknpfungen, die lokale Dateien ffnen
menucolor=black, % Acrobat-Menpunkte
urlcolor=black,
plainpages=false,% zur korrekten Erstellung der Bookmarks
pdfpagelabels,% zur korrekten Erstellung der Bookmarks
hypertexnames=false,% zur korrekten Erstellung der Bookmarks
linktocpage % Seitenzahlen anstatt Text im Inhaltsverzeichnis verlinken
]{hyperref}


% Algorithmen-Formattierung
%\usepackage{algorithm}
\usepackage[final]{listings}
    \lstset{
      breaklines=true,                                     % line wrapping on
%      frame=ltrb,
      framesep=5pt,
      basicstyle=\normalsize,
      stringstyle=\ttfamily,
      showstringspaces=ture
    }
