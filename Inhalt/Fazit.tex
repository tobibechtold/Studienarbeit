\chapter{Fazit \& Ausblick}
\label{cha:Fazit}
\section{Fazit}
In dieser Studienarbeit sollte gezeigt werden, wie eine Webseite die nicht für
Mobilgeräte optimiert ist, auf diesen Geräten angezeigt werden kann. Dazu wurden
mehrere Lösungsmöglichkeiten evaluiert. Die Entscheidung fiel auf die
Entwicklung einer eigenen API, die es ermöglicht die komplex eingebetteten Daten
im readmore.de HTML Quellcode übersichtlich und maschinell verarbeitbar
darzustellen. Desweiteren wurde auf Basis dieser API ein Prototyp einer App
entwickelt, der es ermöglicht das Forum der Webseite komfortabel auf einem
Android Smartphone zu benutzen. Der Protoyp enthält bisher zwar nur grundlegende
Funktionen, wird allerdings auch nach Fertigstellung dieser Arbeit stetig
weiterentwickelt.
\section{Ausblick}
Im Anschluss an diese Arbeit wird der Prototyp der readmore.de Android App in
einer öffentlichen Beta verprobt. Diese Beta dient in erster Linie dem Einholen
von Feedback der späteren Benutzer. Zum einen sollen potentielle Fehler
frühzeitig erkannt werden, zum Anderen können Teilnehmer der Beta ihre Ideen für
fehlende Features einbringen. Zu einem späteren Zeitpunkt ist auch die
Veröffentlichung der App im Google Play Store angedacht. \\
Durch die weite Verbreitung des Json Formats, ist die entwickelte API
plattformunabhängig. Dies lässt den Einsatz auch auf anderen
Plattformen wie iOS oder Windows Phone zu. Daher ist es in Naher Zukunft
denkbar, das Forum von readmore.de auch auf anderen mobilen Betriebssystemen
komfortabel bedienen zu können.
