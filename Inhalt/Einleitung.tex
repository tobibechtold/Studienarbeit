\chapter{Einleitung}
\label{cha:Einleitung}

\section{Einleitung}
Seit dem Jahr 2005 stellt die Esport-Webseite readmore.de Informationen über
alle aktuellen Esport-Titel, laufende Turniere und News über die Esport-Szene
bereit. Ein weiterer großer Teil der Seite stellt das sehr hoch frequentierte
Forum als Diskussionsplattform rund um den Esport dar. Esport bezeichnet dabei
den sportlichen Wettkampf zwischen Menschen mithilfe von Computerspielen. Diese
Wettkämpfe werden meist im Mehrspielermodus der Spiele entweder allein oder in
einer Mannschaft ausgetragen.\footcite{TEST1234} Die aktuell am häufigsten
gespielten und mit den höchsten Preisgeldern dotierten Spiele sind unter anderem \Fachbegriff{Dota 2},
\Fachbegriff{Starcraft 2} und \Fachbegriff{Counter-Strike: Global Offensive}.
Mit Preisgeldern bis zu einer Million US-Dollar pro Turnier zieht alleine der
Esport-Titel \Fachbegriff{Dota 2} über 10 Millionen Spieler weltweit
an.\footnote{Quelle:
http://steamcharts.com/}
Das Forum der Esport-Webseite readmore.de stellt dabei eine der größten
Diskussionsplattformen im deutschsprachigen Raum dar. \\
Allerdings gibt es über 10 Jahre nach der Gründung der Seite noch keine
Möglichkeit, die Seite auf mobilen Endgeräten wie Smartphones oder Tablets in
einer dafür optimierten Ansicht zu betrachten. Die einzige verfügbare Ansicht
ist die Desktopansicht der Seite. Daher sollen in dieser Studienarbeit die
technischen Möglichkeiten untersucht werden, wie diese Seite auch mit mobilen
Endgeräten einfacher betrachtet werden kann. Dies soll mithilfe einer Android
App möglich werden.
\section{Aufgabe und Problemstellung}
Um die readmore.de Webseite auch auf mobilen Endgeräten darzustellen soll eine
Android App entwickelt werden. Diese soll die wichtigste Funktion der Seite, das
Forum, mit nativen Android Komponenten anzeigen. Zusätzlich soll sich der
Benutzer mit seinem readmore.de-Account einloggen und somit auch aktiv im Forum
posten können. Da readmore.de keine API bereitstellt besteht ein wichtiger Teil
der Aufgabe im Auslesen der auf der Webseite angezeigten Daten, wie Foren, Threads und Beiträge. Diese
Aufgabe kann unter Umständen mit einer eigenen Serverkomponente gelöst werden.
Die Kommunikation der App findet dann nicht mehr direkt mit readmore.de statt,
sondern mit einem selbst entwickelten Server der die angeforderte Seite in einem
schnell auslesbaren Format zurückliefert. 
\section{Motivation und Herausforderung}
